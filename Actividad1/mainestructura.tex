\documentclass[12pt]{article}
\usepackage[spanish]{babel}
\usepackage[utf8x]{inputenc}
\usepackage{graphicx}
\usepackage{titlesec}
\usepackage[T1]{fontenc}
\usepackage{setspace}
\titlespacing*{\section}
{0pt}{5.5ex plus 1ex minus .2ex}{4.3ex plus .2ex}
\titlespacing*{\subsection}
{0pt}{5.5ex plus 1ex minus .2ex}{4.3ex plus .2ex}



\begin{document}

%%%%%%%%%%%%%%%%%%%%%%%%%%%%%%%%%%%%%%%%%%%%%%%%%%%%%%%%%%%%%%%%%%%%%%%%%%%%%%
%PORTADA

\begin{titlepage}
\newcommand{\Hrule}{\rule{\linewidth}{0.5mm}}

\begin{center}
\includegraphics[width=4cm]{logo.png}
\end{center}

\begin{center}
\textsc{\LARGE Universidad de Sonora}\\[0.5cm]
\textsc{División de Ciencias Exactas y Naturales}\\[0.1cm]
\textsc{Departamento de Física}\\[1.5cm]
\Hrule \\[0.5cm]   
   \textsc{\Huge \bfseries La estructura de la Atmósfera} \\[0.5cm]
\Hrule \\[1.5cm]
\textsc{\Large Sofía González Montoya} \\[1cm]
\textsc{\Large Profesor: Carlos Lizárraga Celaya} \\[2.5cm]
\textsc{\today}
\end{center}
\end{titlepage}
\pagebreak

%%%%%%%%%%%%%%%%%%%%%%%%%%%%%%%%%%%%%%%%%%%%%%%%%%%%%%%%%%%%%%%%%%%%%%%%%%%%%%
%INTRODUCCIÓN

\doublespacing
\section{\textsc{\Large Resumen}}

En este trabajo se pretende abordar el tema de la estructura de la atmósfera, los fenómenos que ocurren en ella, los parámetros que pueden ser medidos y los instrumentos de medición utilizados para el estudio de los fenómenos físicos ocurridos en la atmósfera. 

\section{\textsc{\Large Introducción}}
El tema a tratar en general es la atmósfera de la Tierra, así como una breve explicación de las cuatro capas componentes, las propiedades tanto físicas como químicas que se producen en estas capas, los fenómenos que ocurren la capa de menor altitud en la que vivimos, llamada tropósfera. También se habla de los diagramas termodinámicos y su continua importancia en el pronóstico de los fenómenos que ocurre en la atmósfera, en particular de cuatro diagramas y sus principales características. \\

A pesar de los avances tecnológicos hasta ahora y las nuevas técnicas de pronósticos se siguen utilizando los diagramas termodinámicos como una herramienta fundamental para el pronóstico del tiempo y existen varios diagramas termodinámicos que permiten analizar los sondeos en los ámbitos de investigación y operativo de las ciencias atmosféricas.

\pagebreak

%%%%%%%%%%%%%%%%%%%%%%%%%%%%%%%%%%%%%%%%%%%%%%%%%%%%%%%%%%%%%%%%%%%%%%%%%%%%%%%
%DESARROLLO DE TEMA

\section{\textsc{\Large Estructura de la atmósfera}}
La atmósfera se define como la envoltura de gases que rodean al planeta Tierra y se compone de cuatro capas en función de su temperatura, composición química, densidad y movimiento: la capa de menor altitud y en la que vivimos es conocida como la tropósfera, que se encuentra en la superficie de la Tierra y es el hogar del fenómeno que afecta a los seres vivos conocido como clima. Seguida está la estratósfera que es donde la capa de ozono reside, después la mesósfera, y por último la termósfera. 


\subsection{\textsc{Las cuatro capas}}
La tropósfera es la capa en la que vivimos y en la cual ocurre el fenómeno conocido como clima, esta capa consta de una altura promedio de 12 km. La altura de la tropósfera varía con la ubicación, siendo más alta en zonas más calientes, con una altura de 17 km en las regiones ecuatoriales y 9 km en los polos. En esta capa la temperatura disminuye cuando la altitud aumenta. Después de la tropósfera se encuentra una región que separa a esta capa de la estratósfera llamada tropopausa en la cual es el nivel más alto que el clima puede suceder, y ahí la temperatura no experimenta variaciones con la altura. \\

En la estratósfera la temperatura aumenta conforme la altura aumenta, diferente a la tropósfera. Aquí se encuentra la capa de ozono, la encargada de absorber los rayos ultravioleta provenientes del Sol e impide que lleguen a la superficie en su forma dañina para los seres vivos. La mesosfera es la más parecida a la tropósfera, pues la temperatura disminuye con la altura y está compuesta por cantidades similares de nitrógeno y oxígeno pero con concentraciones 1000 veces menores y el aire es muy poco denso para que ocurra el clima. La termósfera se extiende hasta los 500-1,000 kilómetros, y a unos 80-550 kilómetros de la superficie terrestre se localiza la ionosfera, en esta capa la temperatura aumenta con la altura por la absorción de una elevada radiación solar.

\section{\textsc{Diagramas termodinámicos}}

Es normal examinar los sondeos atmosféricos durante el proceso de preparación de un pronóstico del tiempo. En la metereología, los diagramas termodinámicos son utilizados para analizar el estado de la atmósfera a través de mediciones obtenidas por radiosondas generalmente obtenidas por globos meteorológicos.

El globo meteorológico es un globo aerostático que lleva instrumentos de medición para suministrar información sobre la atmósfera como presión atmosférica, temperatura y humedad por medio de un aparato de medida llamado radiosonda.Estos globos suelen estar hechos de un material de látex y la radiosonda se coloca en el extremo inferior del globo. Los globos meteorológicos se lanzan en todo el mundo para analizar las condiciones atmosféricas en tiempo real y para crear modelos matemáticos para predicción del tiempo.\\

Para evaluar estos datos de sondeo y la estabilidad atomosférica se utilizan diagramas termodinámicos, y, aún a pesar de los avances en tecnología y en métodos de pronósticos, estos diagramas siguen siendo una herramienta fundamental para el análisis del pronóstico climático.
Estos diagramas normalmente tienen la característica de tener una conexión de 5 lineas: las isobáricas de presión constante representadas por líneas horizontales, isotérmicas de temperatura constante son las líneas diagonales, las adiabáticas secas que son líneas de temperatura potencial constante, las adiabáticas saturadas y las de proporción de mezcla. \\

Existen distintos diagramas termodinámicos, entre ellos están el diagrama Skew-T/log P, el emagrama, Stüve y tefigrama. El tefigrama es empleado para trazar perfiles verticales de temperatura, humedad y viento atmosféricos. Como la presión atmosférica disminuye de forma logarítmica a medida que aumenta la altitud, en este diagrama las líneas isobáricas se trazan en sentido aproximadamente horizontal y van disminuyendo logarítmicamente. El emagrama en su forma original, se utiliza sobre todo en Europa. Al señalar la temperatura y el punto de rocío en emagrama, la estabilidad del aire se puede calcular y el potencial de convección de energía disponible. El diagrama Stüve fue creado en 1927 por Georg Stüve y tiene la simplicidad de que los tres principales parámetros son líneas rectas. 

El diagrama Skew-T/log-P es el que se utiliza desde hace más de 60 años para el pronóstico del tiempo en Estados Unidos y se utiliza para representar perfiles verticales de la atmósfera en forma gráfica. Hace décadas que se utiliza para evaluar una amplia gama de condiciones meteorológicas, principalmente en lo que se refiere a la estabilidad atmosférica. \\

Algunas aplicaciones de pronósticos en las que se utilizan los diagramas termodinámicos son el  pronóstico de temperatura y tipos de precipitación, los cálculos de severidad y potencial de convección, entre otros más. \\

\begin{center}
\includegraphics[width=10cm]{skeewt.png}
\end{center}

\pagebreak

%%%%%%%%%%%%%%%%%%%%%%%%%%%%%%%%%%%%%%%%%%%%%%%%%%%%%%%%%%%%%%%%%%%%%%%%%%%%%%%%%%%%%%%%%%%%%
%BIBLIOGRAFIA

\section{Bibliografía}
Capas de la Atmósfera

http://www.geoenciclopedia.com/capas-de-la-atmosfera/ \\

Structure of the Atmosphere. NC State University

http://climate.ncsu.edu/edu/k12/.AtmStructure \\

Diagramas Termodinámicos

http://www.tutiempo.net/meteorologia/diagramas-termodinamicos.html \\

Skew-T Diagram. MetEd

http://www.meted.ucar.edu/mesoprim/skewt/index.htm

\end{document}
