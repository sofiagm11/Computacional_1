\documentclass[12pt]{article}
\usepackage[spanish]{babel}
\usepackage[utf8x]{inputenc}
\usepackage{graphicx}
\usepackage{setspace}

\begin{document}

\textsc{Preguntas de Reflexión} \\

\begin{enumerate}
\item ¿Cual es tu primera impresión de uso de LaTeX?\\
Pienso que es una plataforma para crear trabajos con calidad y profesionalismo y que será de gran ayuda para los trabajos por realizar y no sólo académicos. 

\item ¿Qué aspectos te gustaron más? \\
Me gustó la forma en que el texto se ve mejor presentado y me gusta ir aprendiendo los comandos que tengo que utilizar para los diferentes aspectos del trabajo.

\item ¿Qué no pudiste hacer en LaTeX?\\
No supe como acomodar las referencias bibliográficas a lo último de mi trabajo


\item En tu experiencia, comparado con otros editores, ¿cómo se compara LaTeX? \\
Latex te ayuda a presentar un trabajo con una mejor calidad y de manera más profesional.

\item ¿Qué es lo que mas te llamó la atención en el desarrollo de esta actividad? \\
El entender los comandos necesarios para elaborar esta actividad y que el trabajo se vea presentable.

\item ¿Qué cambiarías en esta actividad? \\
Pediría ayuda para poner la bibliografía de manera correcta.

\item ¿Que consideras que falta en esta actividad? \\
Quizá haber profundizado un poco más en los instrumentos de medición de los fenómenos que ocurren en la atmósfera.

\item ¿Puedes compartir alguna referencia nueva que consideras útil y no se haya contemplado?
Del mismo link para el diagrama Skew-T, también para los demás diagramas mencionados en el trabajo

\item ¿Algún comentario adicional que desees compartir? \\
A pesar de ya haber usado latex en otros trabajos principalmente reportes de laboratorio, se me sigue haciendo difícil aprender todos los comandos pero los tutoriales siempre te sacan del apuro.

\end{enumerate}

\end{document}
